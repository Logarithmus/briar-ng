%! TEX root = course_report.tex

\section{Платформа Android}
\label{sec:android_platform}

Первая версия операционной системы Android была представлена более 15 лет назад.
Вскоре после появления Android, права на эту ОС купила корпорация Google, увидев в ней огромный потенциал.
В результате на сегодняшний день Android -- наиболее популярная ОС для мобильных устройств с рыночной долей в 71\%, по данным StatCounter \cite{statcounter_mobile_os}.
StatCounter -- сервис веб-аналитики, установленный на более чем 2 миллионах сайтов по всему миру.
Отчёты StatCounter основываются на данных, собранных с помощью данного сервиса \cite{statcounter_methodology}.

Говоря об Android, обычно подразумевают платформу Android -- более широкое понятие, включающее в себя, помимо слоя абстракции оборудования (англ. \textit{Hardware Abstraction Layer}) и ядра, различные высокоуровневые средства для создания пользовательских приложений на языках Java и Kotlin.

\subsection{Структура и архитектура платформы}
\label{sub:android_platform:struct_and_arch}

Платформа Android состоит из нескольких слоёв абстракции, разделённых между собой для упрощения разработки каждого из них: 

1 \textit{Ядро Linux} лежит в основе платформы Android.
Одна из причин использования ядра Linux -- широкие возможности Linux, нацеленные на повышение безопасности \cite{android_kernel_security}:
\begin{itemize}
	\item безопасное межпроцессное взаимодействие;
	\item изоляция приложений друг от друга;
	\item система прав доступа к файлам и папкам;
	\item защита загрузчика от несанкционированного изменения;
	\item современные криптографические алгоритмы.
\end{itemize}
Вместе со сборкой ядра Linux производители мобильных устройств поставляют драйверы оборудования (обычно в виде модулей с закрытым исходным кодом).

2 \textit{Слой абстракции оборудования (HAL)} работает поверх ядра.
Он содержит библиотеки функций для более удобного доступа к оборудованию различных производителей из вышестоящих уровней абстракции.
HAL состоит из нескольких модулей, каждый из которых реализует интерфейс для определённого типа оборудования: камера, Bluetooth-модуль, микрофон, звуковой динамик и т. д. 

3 \textit{Android Runtime (ART)} -- среда выполнения Java и Kotlin приложений.
Она осуществляет компиляцию байт-кода ART в машинный код процессора, установленного в пользовательском устройстве.
Android Runtime пришёл на cмену Dalvik в версии Android 5.0.
Одним из преимуществ Android Runtime по сравнению с Dalvik является компиляция приложений во время установки в машинный код устройства (англ. \textit{ahead-of-time}).
Это позволяет значительно сократить время запуска приложения.
При этом установка приложений может занимать больше времени, чем при компиляции во время выполнения (англ. \textit{just-in-time}).

На одном уровне с \textit{ART} находятся \textit{системные C и C++ библиотеки}, предоставляющие высокоуровневый интерфейс для распространённых задач:
\begin{itemize}
	\item отображениe веб-страниц (Android System WebView);
	\item стандартная библиотека языка C (Bionic);
	\item работа с графикой (OpenGL ES и Vulkan);
	\item работа со звуком (OpenSL ES) и др.
\end{itemize}

При разработке приложений с использованием Android NDK возможно напрямую работать с вышеописанными библиотеками.

4 \textit{Java API Framework} оборачивает \textit{системные библиотеки} и \textit{слой абстракции оборудования} в Java-классы, которые используются в системных и пользовательских приложениях. В состав Java API Framework входят:
\begin{itemize}
	\item \textit{View System} -- расширяемая система компонентов пользовательского интерфейса (кнопки, списки, поля ввода текста и т. д.);
	\item \textit{Resource Manager} -- управление файлами локализации, графики и описания пользовательского интерфейса;
	\item \textit{Notification Manager} -- для отображения оповещений в строке состояния (т. н. <<шторка>>);
	\item \textit{Activity Manager}, управляющий жизненным циклом приложения и навигацией между экранами;
	\item \textit{Content Providers}, служащие для доступа к данным из других приложений (например, к списку контактов) и др.
\end{itemize}

5 \textit{Стандартные системные приложения} составляют самый высокий уровень абстракции и включают:
\begin{itemize}
	\item приложение для совершения звонков по мобильной сети (\textit{Dialer});
	\item приложение для отправки SMS;
	\item клиент электронной почты;
	\item календарь;
	\item камера и др.
\end{itemize}

На рисунке~\ref{fig:android_platform_arch} наглядно представлены компоненты платформы Android.

\begin{figure}[p]
    \centering
    \includegraphics[width=1.0\textwidth]{android_platform_arch.png}  
    \caption{Компоненты платформы Android}
	\label{fig:android_platform_arch}
\end{figure}

\subsection{История, версии и достоинства Android}
\label{sub:android_platform:history_and_pros}


