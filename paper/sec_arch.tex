%! TEX root = course_report.tex

\section{Архитектура мобильного приложения} % (fold)
\label{sec:arch}


\subsection{Общая структура приложения}
\label{sub:arch:struct}

Разработанное программное обеспечение представляет из себя библиотеку кода написанную на языках \fsharp{} и \csharp{}.

Так как в курсовом проекте рассматривается одна из разновидностей графовых моделей, то, очевидно, для представления таких моделей в разрабатываемой библиотеке должна быть часть, отвечающая за представление и работу с графами.
При реализации здесь было несколько альтернативных путей: использовать одну из доступных библиотек для платформы \dotnet{} для работы с графами или реализовать собственную.
Среди готовых библиотек можно было бы использовать QuickGraph, Directed Graph for .NET или GrapheNET.
Но было принято решение остановиться на варианте, подразумевающем разработку собственных типов для работы с графами.

Одним из важных решений, которое было принято в начале проектирования модуля работы с графами, было использование по возможности неизменяемых структур данных.
Это решение выгодно отличает разработанную реализацию от существующих библиотек, от использования которых было принято решение отказаться. 
Существующие библиотеки ориентированы на работу в императивном стиле и с изменяемым состоянием.
Также использование неизменяемых структур данных для реализации типов для представления графов в дальнейшем положительно сказалось на простоте реализации поиска структуры вероятностной сети в алгоритмах вывода структуры по данным.
Часть разрабатываемой библиотеки, содержащая типы для работы с графами, реализована на языке программирования \fsharp{}.

