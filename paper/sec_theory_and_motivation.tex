%! TEX root = course_report.tex

\section{Теоретическое обоснование разработки программного продукта} 
\label{sec:theory_and_motivation}

Выбор технологий является важным предварительным этапом разработки сложных информационных систем.
Платформа и язык программирования, на котором будет реализована система, заслуживает большого внимания, так как исследования показали, что выбор языка программирования влияет на производительность труда программистов и качество создаваемого ими кода~\cite[c.~59]{mcconnell_2005}.

Ниже перечислены некоторые факторы, повлиявшие на выбор технологий:
\begin{itemize}
\item Разрабатываемое ПО должно работать на операционной системе Windows~7 и более новых версиях системы.
\item Среди различных платформ разработки имеющийся программист лучше всего знаком с разработкой на платформе \dotnet{}.
\item Дальнейшей поддержкой проекта, возможно, будут заниматься разработчики, не принимавшие участие в выпуске первой версии.
\item Имеющийся разработчик имеет опыт работы с объекто"=ориентированными и с функциональными языками программирования.
\end{itemize}

Основываясь на опыте работы имеющихся программистов разрабатывать ПО целесообразно на платформе \dotnet{}.
Приняв во внимание необходимость обеспечения доступности дальнейшей поддержки ПО, возможно, другой командой программистов, целесообразно не использовать малоизвестные и сложные языки программирования.
С учетом этого фактора выбор языков программирования сужается до четырех официально поддерживаемых Microsoft и имеющих изначальную поддержку в Visual Studio~2012: \cppcli{}, \csharp{}, \vbnet{} и \fsharp{}.
Необходимость использования низкоуровневых возможностей \cppcli{} в разрабатываемом ПО отсутствует, следовательно данный язык можно исключить из списка кандидатов.
\vbnet{} уступает по удобству использования двум другим кандидатам из нашего списка.
Оставшиеся два языка программирования \csharp{} и \fsharp{} являются первостепенным, элегантными, мультипарадигменными языками программирования для платформы \dotnet.
Таким образом, с учетом вышеперечисленных факторов, целесообразно остановить выбор на следующих технологиях:
\begin{itemize}
  \item операционная система Windows~7;
  \item платформа разработки \dotnet{};
  \item языки программирования \csharp{} и \fsharp{}.
\end{itemize}
Для реализации поставленной задачи нет необходимости в использовании каких"=либо прикладных библиотек для создания настольных или веб"=приложений, достаточно использовать стандартные библиотеки указанных выше языков.
Поддержка платформой \dotnet{} различных языков программирования позволяет использовать язык, который наиболее просто и <<красиво>> позволяет решить возникающую задачу.
Разрабатываемое программное обеспечение в некоторой степени использует данное преимущество платформы.
Язык \csharp{} больше подходит для создания высокоуровнего дизайна проложения (иерархия классов и интерфейсов, организация пространств имен и публичного программного интерфейса), язык \fsharp{} "--- для реализации логики приложения, функций и методов~\cite{fsdg_2010}, прототипирования различных идей.
В разрабатываемом программном продукте \csharp{} используется для предоставления удобного программного интерфейса, \fsharp{} "--- для прототипирования и реализации вычислительной логики.
Далее проводится характеристика используемых технологий и языков программирования более подробно.

\subsection{Обоснование необходимости разработки приложения}
\label{sub:theory_and_motivation:motivation}
Программная платформа \dotnet{} является одной из реализаций стандарта ECMA-335~\cite{ecma_335} и является современным инструментом создания клиентских и серверных приложений для операционной системы Windows.
Первая общедоступная версия \netfx{} вышла в феврале 2002 года.
С тех пор платформа активно эволюционировала и на данный момент было выпущено шесть версии данного продукта.
На данный момент номер последней версии \netfx{} "--- 4.5.
Платформа \dotnet{} была призвана решить некоторые наболевшие проблемы, скопившиеся на момент её выхода, в средствах разработки приложений под Windows. 
Ниже перечислены некоторые из них~\cite[с.~\Rmnum{14}\,--\,\Rmnum{17}]{richter_2007_ru}:
\begin{itemize}
  \item сложность создания надежных приложений;
  \item сложность развертывания и управления версиями приложений и библиотек;
  \item сложность создания переносимого ПО;
  \item отсутствие единой целевой платформы для создателей компиляторов;
  \item проблемы с безопасным исполнением непроверенного кода;
  \item великое множество различных технологий и языков программирования, которые не совместимы между собой.
\end{itemize}

Многие из этих проблем были решены.
Далее более подробно рассматривается внутреннее устройство \dotnet{}.

Основными составляющими компонентами \dotnet{} являются общая языковая исполняющая среда (Common Language Runtime) и стандартная библиотека классов (Framework Class Library).
CLR представляет из себя виртуальную машину и набор сервисов обслуживающих исполнение программ написанных для \dotnet{}.
Ниже приводится перечень задач, возлагаемых на CLR~\cite{marchenko_2007}:
\begin{itemize}
  \item загрузка и исполнение управляемого кода;
  \item управление памятью при размещении объектов;
  \item изоляция памяти приложений;
  \item проверка безопасности кода;
  \item преобразование промежуточного языка в машинный код;
  \item доступ к расширенной информации от типах "--- метаданным;
  \item обработка исключений, включая межъязыковые исключения;
  \item взаимодействие между управляемым и неуправляемым кодом (в том числе и COM"=объектами);
  \item поддержка сервисов для разработки (профилирование, отладка и т.\,д.).
\end{itemize}

Программы написанные для \dotnet{} представляют из себя набор типов взаимодействующих между собой.
\dotnet{} имеет общую систему типов (Common Type System, CTS).
Данная спецификация описывает определения и поведение типов создаваемых для \dotnet{}~\cite{richter_2012_en}.
В частности в данной спецификации описаны возможные члены типов, механизмы сокрытия реализации, правила наследования, типы"=значения и ссылочные типы, особенности параметрического полиморфизма и другие возможности предоставляемые CLI.
Общая языковая спецификация (Common Language Specification, CLS) "--- подмножество общей системы типов. 
Это набор конструкций и ограничений, которые являются руководством для создателей библиотек и компиляторов в среде \netfx{}.
Библиотеки, построенные в соответствии с CLS, могут быть использованы из любого языка программирования, поддерживающего CLS. 
Языки, соответствующие CLS (к их числу относятся языки \csharp{}, \vbnet{}, \cppcli{}), могут интегрироваться друг с другом. CLS "--- это основа межъязыкового взаимодействия в рамках платформы \dotnet{}~\cite{marchenko_2007}.

Некоторые из возможностей, предоставляемых \dotnet{}: верификация кода, расширенная информация о типах во время исполнения, сборка мусора, безопасность типов, "--- невозможны без наличия подробных метаданных о типах из которых состоит исполняемая программа.
Подробные метаданные о типах генерируются компиляторами и сохраняются в результирующих сборках.
Сборка "--- это логическая группировка одного или нескольких управляемых модулей или файлов ресурсов, является минимальной единицей с точки зрения повторного использования, безопасности и управлениями версиями~\cite[с.~6]{richter_2012_en}.

Одной из особенностей \dotnet{}, обеспечивающей переносимость программ без необходимости повторной компиляции, является представление исполняемого кода приложений на общем промежуточном языке (Common Intermediate Language, CIL). 
Промежуточный язык является бестиповым, стековым, объекто"=ориентированным ассемблером~\cite[с.~16\,--\,17]{richter_2012_en}.
Данный язык очень удобен в качестве целевого языка для создателей компиляторов и средств автоматической проверки кода для платформы \dotnet{}, также язык довольно удобен для чтения людьми.
Наличие промежуточного языка и необходимость создания производительных программ подразумевают наличие преобразования промежуточного кода в машинный код во время исполнения программы.
Одним из компонентов общей языковой исполняющей среды, выполняющим данное преобразование, является компилятор времени исполнения (Just-in-time compiler) транслирующий промежуточный язык в машинные инструкции, специфические для архитектуры компьютера на котором исполняется программа.

Ручное управление памятью всегда являлось очень кропотливой и подверженной ошибкам работой.
Ошибки в управлении памятью являются одними из наиболее сложных в устранении типами программных ошибок, также эти ошибки обычно приводят к непредсказуемому поведению программы, поэтому в \dotnet{} управление памятью происходит автоматически~\cite[с.~505\,--\,506]{richter_2012_en}.
Автоматическое управление памятью является механизмом поддержания иллюзии бесконечности памяти.
Когда объект данных перестает быть нужным, занятая под него память автоматически освобождается и используется для построения новых объектов данных.
Имеются различные методы реализации такого автоматического распределения памяти~\cite[с.~489]{sicp_2006_ru}.
В~\dotnet{} для автоматического управления памятью используется механизм сборки мусора (garbage collection).
Существуют различные алгоритмы сборки мусора со своими достоинствами и недостатками. 
В \dotnet{} используется алгоритм пометок (mark and sweep) в сочетании с различными оптимизациями, такими как, например, разбиение всех объектов по поколениям и использование различных куч для больших и малых объектов.

Ниже перечислены, без приведения подробностей, некоторые важные функции исполняемые общей языковой исполняющей средой:
\begin{itemize}
  \item обеспечение многопоточного исполнения программы;
  \item поддержание модели памяти, принятой в CLR;
  \item поддержка двоичной сериализации;
  \item управление вводом и выводом;
  \item структурная обработка исключений;
  \item возможность размещения исполняющей среды внутри других процессов.
\end{itemize}

Как уже упоминалось выше, большую ценностью для \dotnet{} представляет библиотека стандартных классов "--- соответствующая CLS"=спецификации объектно"=ориентированная библиотека классов, интерфейсов и системы типов (типов"=значений), которые включаются в состав платформы \dotnet{}.
Эта библиотека обеспечивает доступ к функциональным возможностям системы и предназначена служить основой при разработке .NET"=приложений, компонент, элементов управления~\cite{marchenko_2007}.



\subsection{Технологии программирования, используемые для решения поставленных задач}
\label{sub:theory_and_motivation:tools}


\subsubsection{Язык программирования Java}
\label{subsub:theory_and_motivation:tools:java}

\subsubsection{Язык программирования Kotlin}
\label{subsub:theory_and_motivation:tools:kotlin}

\subsubsection{Android API Framework}
\label{subsub:theory_and_motivation:tools:android_api_framework}

\subsubsection{Система сборки Gradle}
\label{subsub:theory_and_motivation:tools:gradle}

\subsubsection{Интегрированная среда разработки IntelliJ IDEA}
\label{subsub:theory_and_motivation:tools:intellij_idea}

\subsubsection{Текстовый редактор NeoVIM}
\label{subsub:theory_and_motivation:tools:neovim}

\subsubsection{Интегрированная среда разработки IntelliJ IDEA}
\label{subsub:theory_and_motivation:tools:intellij_idea}
