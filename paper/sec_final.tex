%! TEX root = course_report.tex

\sectioncentered*{Заключение}
\addcontentsline{toc}{section}{Заключение}

В ходе работы над данным курсовым проектом был разработан прототип приложения для децентрализованного обмена сообщениями с устройствами поблизости. Отличительной особенностью приложения является возможность работы без подключения к Интернету либо при блокировках отдельных ресурсов сети Интернета со стороны контролирующих органов.

Целевая аудитория приложения -- активисты, журналисты и простые граждане в странах с ограничениями свободы слова. Также разработанный продукт может быть полезен при стихийных бедствиях для координации помощи пострадавшим людям при проблемах в работе телекоммуникаций.

В результате разработки были реализованы следующие функции:
\begin{itemize}
	\item связь с устройствами поблизости по Bluetooth;
	\item передача и получение текстовых сообщений;
	\item отправка и приём изображений;
	\item отправка установочного файла приложения другим людям;
	\item светлая и тёмная темы оформления.
\end{itemize}

К сожалению, при попытках реализовать сквозное шифрование с помощью привязок к библиотеке libsodium и работу через Wi-Fi Direct возникли определённые трудности:
\begin{itemize}
	\item сложность работы с API Wi-Fi Direct, предоставляемым Android;
	\item нехватка времени;
	\item проблемы с версией Android NDK, используемой libsodium и др.
\end{itemize}
