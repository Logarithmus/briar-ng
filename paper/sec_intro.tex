%! TEX root = course_report.tex

\sectioncentered*{Введение}
\addcontentsline{toc}{section}{Введение}
\label{sec:intro}

Мобильные устройства и Интернет прочно вошли в нашу жизнь.
На сегодняшний день практически все мобильные телефоны являются смартфонами: кроме совершения голосовых вызовов, с их помощью можно подключаться к сети Интернет, обмениваться файлами, слушать музыку, ориентироваться на местности и т. д.

В последние годы стали набирать популярность мессенджеры.
Изначально под словом <<мессенджер>> понималось приложение для мгновенного (очень быстрого) обмена короткими сообщениями, как правило, посредством сети Интернет.
Со временем в мессенджерах появилась поддержка голосовых и видеовызовов, передачи файлов, создания открытых и закрытых чатов.
Некоторые мессенджеры позволяют создавать видеоконференции с участием большого количества людей, а также демонстрировать содержимое экрана другим участникам конференции.

Последним трендом в развитии мессенджеров стало внедрение сквозного шифрования.
С технической точки зрения сквозное шифрование обычно реализуется с помощью алгоритмов асимметричной криптографии (например, RSA) и обмена публичными ключами по протоколу Диффи-Хеллмана. Сквозное шифрование позволяет добиться того, что содержимое сообщения доступно только отправителю и получателю.
Данное свойство позволяет значительно повысить приватность и безопасность общения, поскольку ни создатели мессенджера, ни спецслужбы, ни злоумышленники, взломавшие сервер, не могут расшифровать сообщение.

Несмотря на большой прогресс в развитии приложений для мгновенного общения, большинство таких приложений требует постоянного подключения к Интернету и соединения с центральным сервером или серверным кластером.
Поэтому следующим логичным шагом в эволюции мессенджеров в сторону безопасности и надёжности является децентрализация.
Децентрализация подразумевает отсутствие центрального сервера или серверного кластера, в результате чего пропадает единая точка отказа.
Чтобы вывести из строя такую систему, необходимо получить контроль над всеми её узлами.

В рамках данного курсового проекта был разработан прототип приложения для децентрализованного обмена сообщениями в условиях, когда доступ в Интернет ограничен либо отсутствует вовсе.
Приложение работает на базе технологий Bluetooth и Wi-Fi Direct, поддержка которых есть в абсолютном большинстве смартфонов, выпущенных за последние пять лет.
